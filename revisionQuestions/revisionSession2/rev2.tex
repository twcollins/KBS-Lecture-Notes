
\documentclass[%
pdf,
%nocolorBG,
colorBG,
slideColor,
tcrico,
%slideBW,
%draft,
%frames
%azure
%contemporain
%nuancegris
%troispoints
%lignesbleues
%darkblue
%alienglow
%autumn
%default
%gyom
%blends
]{prosper}
\usepackage{amsmath}
%\usepackage[table]{xcolor}
\usepackage{colortbl}
\usepackage{graphicx}
\usepackage{subfigure}
\usepackage{algorithm2e}

\begin{document}


\begin{slide}{Knowledge Based Technologies}
	\textbf{Revision Session 2}

	\small
	John Moore \& Thomas Collins
\end{slide}

%%%%%%%%%%%%%%%%%%%%%%%%%%%%%%%%%%%%%%%%%%%%%%%%%%%%%%%%%%%%%%%%%%%%%%%%%%%%%%%%%%%%%%%%%%%%%%%%%%%%%%%%%%%
\begin{slide}{Topics} 

\begin{enumerate}
\item Bayesian Learning
\item Reinforcement Learning
\item Genetic Algorithms
\item Instance Based Learning	
\item Analytical Learning
\end{enumerate}
\end{slide}

%%%%%%%%%%%%%%%%%%%%%%%%%%%%%%%%%%%%%%%%%%%%%%%%%%%%%%%%%%%%%%%%%%%%%%%%%%%%%%%%%%%%%%%%%%%%%%%%%%%%%%%%%%%
\begin{slide}{Bayesian Learning} 
\tiny
\begin{itemize}
\item What does Bayes theorem state?
\item Describe the meaning of Bayes theorem in words i.e. natural language?
\item State using appropriate notation, such as that encountered during the associated lecture, Bayes theorem.
\item What do the individual components of Bayes theorem state.
\item In the context of Bayesian learning what is a the maximum a posteriori (MAP) hypothesis?
\item How would you calculate a MAP hypothesis using Bayes theorem?
\item What is a maximum likelihood (ML) hypothesis?
\item May a ML hypothesis ever be identical to a MAP hypothesis. If so under what conditions, if not why not?
\end{itemize}
\end{slide}


%%%%%%%%%%%%%%%%%%%%%%%%%%%%%%%%%%%%%%%%%%%%%%%%%%%%%%%%%%%%%%%%%%%%%%%%%%%%%%%%%%%%%%%%%%%%%%%%%%%%%%%%%%%
\begin{slide}{Reinforcement Learning} 
\tiny
\begin{itemize}
\item What is the underlying basis of Reinforcement Learning (RL)?
\item What are the constitutive elements of a RL problem?
\item What is a 'Reward Function'?
\item What is a 'Value Function'?
\item How do a 'Reward Function' and a 'Value Function' relate to each other?
\item Can you provide a synopsis of Temporal Difference techniques such as Q Learning?  In particular:
	\begin{itemize}
	\item General description
	\item Pseudo code algorithm
	\item In the context of the technique how does the agent represent it's knowledge of the world?
	\item What information does the agent need to update it's current beliefs (model). 
	\item Are TD methods different from Monte Carlo methods? If so in what way, if not why not?
	\end{itemize}
\end{itemize}
\end{slide}

%%%%%%%%%%%%%%%%%%%%%%%%%%%%%%%%%%%%%%%%%%%%%%%%%%%%%%%%%%%%%%%%%%%%%%%%%%%%%%%%%%%%%%%%%%%%%%%%%%%%%%%%%%%

%%%%%%%%%%%%%%%%%%%%%%%%%%%%%%%%%%%%%%%%%%%%%%%%%%%%%%%%%%%%%%%%%%%%%%%%%%%%%%%%%%%%%%%%%%%%%%%%%%%%%%%%%%%
\begin{slide}{Genetic Algorithms} 
\tiny
\begin{itemize}
\item What is the underlying premise of Genetic Algorithms (GA)?
\item Explain what the following terms mean in the context of GAs
	\begin{itemize}
	\item Chromosomes, Genes, Alleles, Fitness, Selection, Crossover, Mutation
	\end{itemize}
\item Outline, using pseudo code, a simple GA.
\item What is a schema in the context of GAs?
\item What is a schema qualifier?
\item What is the schema theorem and what does it state?
\end{itemize}
\end{slide}

%%%%%%%%%%%%%%%%%%%%%%%%%%%%%%%%%%%%%%%%%%%%%%%%%%%%%%%%%%%%%%%%%%%%%%%%%%%%%%%%%%%%%%%%%%%%%%%%%%%%%%%%%%%
\begin{slide}{Analytical Learning} 
\begin{itemize}
\item What is analytical  learning and how does it differ from inductive learning?
\item What role does 'prior knowledge' play in the context of Analytical Learning?
\item What is the Analytical Generalisation Problem?
\item What is a perfect domain theory?
\end{itemize}
\end{slide}

%%%%%%%%%%%%%%%%%%%%%%%%%%%%%%%%%%%%%%%%%%%%%%%%%%%%%%%%%%%%%%%%%%%%%%%%%%%%%%%%%%%%%%%%%%%%%%%%%%%%%%%%%%%
\begin{slide}{Exam Structure} 
\begin{itemize}
\item Two parts A \& B. 
	\begin{itemize}
	\item Part A: Prolog questions. Approximately 8. Answer all.
	\item Part B: ML questions. Answer 2 from 3. 
	\end{itemize}
\item Both parts A and B are compulsory.
\end{itemize}
\end{slide}




\end{document}




