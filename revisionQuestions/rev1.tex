
\documentclass[%
pdf,
%nocolorBG,
colorBG,
slideColor,
tcrico,
%slideBW,
%draft,
%frames
%azure
%contemporain
%nuancegris
%troispoints
%lignesbleues
%darkblue
%alienglow
%autumn
%default
%gyom
%blends
]{prosper}
\usepackage{amsmath}
%\usepackage[table]{xcolor}
\usepackage{colortbl}
\usepackage{graphicx}
\usepackage{subfigure}
\usepackage{algorithm2e}

\begin{document}


\begin{slide}{Knowledge Based Technologies}
	\textbf{Revision Session 1}

	\small
	John Moore \& Thomas Collins
\end{slide}

%%%%%%%%%%%%%%%%%%%%%%%%%%%%%%%%%%%%%%%%%%%%%%%%%%%%%%%%%%%%%%%%%%%%%%%%%%%%%%%%%%%%%%%%%%%%%%%%%%%%%%%%%%%
\begin{slide}{Topics} 

\begin{enumerate}
\item Artificial Intelligence \& Machine Learning
\item Induction Learning
\item Decision Tree Learning
\item Artificial Neural Networks
\end{enumerate}
\end{slide}

%%%%%%%%%%%%%%%%%%%%%%%%%%%%%%%%%%%%%%%%%%%%%%%%%%%%%%%%%%%%%%%%%%%%%%%%%%%%%%%%%%%%%%%%%%%%%%%%%%%%%%%%%%%
\begin{slide}{Artificial Intelligence \& Machine Learning} 
General considerations?
\begin{itemize}
\item What do we mean by learning?
\item What, if any, clues can we derive from biological learning systems?
\item What is a well defined learning problem?
\item How does the number of training examples influence accuracy?
\item What do we mean by noisy data and what role does it play in the learning process?
\item What is prior knowledge and what role can it play in the learning process?
\end{itemize}
\end{slide}


%%%%%%%%%%%%%%%%%%%%%%%%%%%%%%%%%%%%%%%%%%%%%%%%%%%%%%%%%%%%%%%%%%%%%%%%%%%%%%%%%%%%%%%%%%%%%%%%%%%%%%%%%%%
\begin{slide}{Induction Learning} 
\begin{itemize}
\item What do we mean by the term 'Hypotheses'?
\item What does the 'Inductive Learning Hypothesis' state?
\item What does the term 'Bias' mean in the context of a 'Machine Learning' algorithm/technique?
\item Some other biases we talked about were 'Restriction Bias' and 'Preference Bias'. What is the difference between both?
\item What is meant by the term 'Inductive Bias of a learner'?
\item Some learning techniques we encountered were 'Find S', 'Candidate Elimination'. What are their respective inductive biases?
\end{itemize}
\end{slide}

%%%%%%%%%%%%%%%%%%%%%%%%%%%%%%%%%%%%%%%%%%%%%%%%%%%%%%%%%%%%%%%%%%%%%%%%%%%%%%%%%%%%%%%%%%%%%%%%%%%%%%%%%%%

%%%%%%%%%%%%%%%%%%%%%%%%%%%%%%%%%%%%%%%%%%%%%%%%%%%%%%%%%%%%%%%%%%%%%%%%%%%%%%%%%%%%%%%%%%%%%%%%%%%%%%%%%%%
\begin{slide}{Decision Tree Learning} 
\begin{itemize}
\item What does the term 'Entropy' mean?
\item What does the term 'Information Gain' mean?
\item What is the general inductive bias of decision tree learning?
\item What does 'Over Fitting' generally mean in the context of DT learning?
\end{itemize}
\end{slide}
%%%%%%%%%%%%%%%%%%%%%%%%%%%%%%%%%%%%%%%%%%%%%%%%%%%%%%%%%%%%%%%%%%%%%%%%%%%%%%%%%%%%%%%%%%%%%%%%%%%%%%%%%%%

%%%%%%%%%%%%%%%%%%%%%%%%%%%%%%%%%%%%%%%%%%%%%%%%%%%%%%%%%%%%%%%%%%%%%%%%%%%%%%%%%%%%%%%%%%%%%%%%%%%%%%%%%%%
\begin{slide}{ANN's} 
\tiny
\begin{itemize}
\item What are the main principles of Backpropagation learning (BPL)?
\item What are the key concepts contained within the BPL  paradigm?
\item What are the key objectives of BPL?
\item In the general sense which of the following structural assumptions has the greatest affect on the trade off between under-fitting and overfitting in a NN?
\begin{itemize}
\item The learning rate
\item The initial choice of weights
\item The number of hidden nodes
\item The use of a constant term unit input
\end{itemize}
\item Using appropriate 'pseudo code' outline the general BPL algorithm?
\item How does BPL appropriate a real valued function?
\end{itemize}
\end{slide}
%%%%%%%%%%%%%%%%%%%%%%%%%%%%%%%%%%%%%%%%%%%%%%%%%%%%%%%%%%%%%%%%%%%%%%%%%%%%%%%%%%%%%%%%%%%%%%%%%%%%%%%%%%%



\end{document}




