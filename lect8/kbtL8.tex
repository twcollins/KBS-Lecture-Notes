
\documentclass[%
pdf,
%nocolorBG,
colorBG,
slideColor,
tcrico,
%slideBW,
%draft,
%frames
%azure
%contemporain
%nuancegris
%troispoints
%lignesbleues
%darkblue
%alienglow
%autumn
%default
%gyom
%blends
]{prosper}
\usepackage{amsmath}
%\usepackage[table]{xcolor}
\usepackage{colortbl}
\usepackage{graphicx}
\usepackage{subfigure}
\usepackage{algorithm2e}

\begin{document}


%\begin{slide}{Knowledge Based Technologies}
%Working notes.
%Misc Material:
%\begin{itemize}
% \item  http://jmvidal.cse.sc.edu/talks/analyticallearning/allslides.html
%  \item  
%\end{itemize}
%\end{slide}



\begin{slide}{Knowledge Based Technologies}
	\textbf{Lecture 8} 
	\newline
	\textbf{An Introduction to Analytical Learning}

	\small
	John Moore \& Thomas Collins
\end{slide}

%%%%%%%%%%%%%%%%%%%%%%%%%%%%%%%%%%%%%%%%%%%%%%%%%%%%%%%%%%%%%%%%%%%%%%%%%%%%%%%%%%%%%%%%%%%%%%%%%%%%%%%%%%%

\begin{slide}{Overview}
\tiny
		\begin{itemize}
		\item To date we have studied inductive learning methods (learning from examples).
		\item Induction fails when there is very little data. 
			\begin{itemize}
			\item If the hypothesis space $H$ is finite, and $D$ is a sequence of $m \geq 1$ independent random examples of some target concept there exists a verifiable and bounded probability that the learner will output an erroneous hypothesis
    			\item That is, any consistent learner can PAC-learn any target concept in $H$ with $m$ examples, where PAC-learn means that the hypothesis is probably ($1 - \delta$) approximately (within error ε) correct.
			\end{itemize}
		\item Can we address that bound?
    		\item Yes, if we re-state the learning problem
		\item Explanation-based learning \dots
		\end{itemize}
\end{slide}

%%%%%%%%%%%%%%%%%%%%%%%%%%%%%%%%%%%%%%%%%%%%%%%%%%%%%%%%%%%%%%%%%%%%%%%%%%%%%%%%%%%%%%%%%%%%%%%%%%%%%%%%%%%

\begin{slide}{Explanation Based Learning}
\tiny
	\begin{itemize}
	 \item Learning algorithm accepts explicit prior knowledge as an input, in addition to the training data.
    	\item Inverted deduction systems also use background knowledge, but they use it to augment the description of instances. \[\forall_{\langle x_{i}, f(x_i) \rangle \in D} B \wedge h \wedge x_{i} \models f(x_{i}) \] This results in increasing the size of H.
    	\item In explanation-based learning the prior knowledge is used to reduce the size of H. EBL assumes that \[ \forall_{\langle x_{i}, f(x_i) \rangle \in D} B' \wedge x_{i} \models f(x_{i}) \] and outputs $h$ such that \[ \forall_{\langle x_{i}, f(x_i) \rangle \in D} h \wedge x_{i} \models f(x_{i}) \]\[ D \wedge B' \models h \]
	\end{itemize}

\end{slide}


%%%%%%%%%%%%%%%%%%%%%%%%%%%%%%%%%%%%%%%%%%%%%%%%%%%%%%%%%%%%%%%%%%%%%%%%%%%%%%%%%%%%%%%%%%%%%%%%%%%%%%%%%%%

\begin{slide}{An Example }
	\begin{figure}
		\centering
		\includegraphics[scale=0.35]{./../bookps/chess.ps}
	\end{figure}
\tiny
	\begin{itemize}
	 \item  Want program to recognise "chessboard positions in which black will lose its queen within two moves."
    \item Because there are so many possible chessboards we would nee to provide a lot of examples.
    \item And yet, humans can learn this concept really quickly. Why?
    \item Humans appear to rely heavily on explaining the training example in terms of their prior knowledge.

	\end{itemize}

\end{slide}


%%%%%%%%%%%%%%%%%%%%%%%%%%%%%%%%%%%%%%%%%%%%%%%%%%%%%%%%%%%%%%%%%%%%%%%%%%%%%%%%%%%%%%%%%%%%%%%%%%%%%%%%%%%

\begin{slide}{Explanations}
\tiny
	\begin{itemize}
	\item The explanation tells us how to rationally  generalise from the details of the correct training example to a correct general hypothesis.
	\item  Namely, the features of the training example that are mentioned in the explanation are the ones that should be included in the hypothesis.
\item What is the prior knowledge needed in the chess example?
\item The knowledge about the legal moves of chess, the fact that players alternate moves, and the goal is to capture the king.
\item Note that, in principle we could use this knowledge to calculate the optimal move for every board, but this is not practical in most situations
	\end{itemize}
\end{slide}


%%%%%%%%%%%%%%%%%%%%%%%%%%%%%%%%%%%%%%%%%%%%%%%%%%%%%%%%%%%%%%%%%%%%%%%%%%%%%%%%%%%%%%%%%%%%%%%%%%%%%%%%%%%


\begin{slide}{Inductive vs. Analytical Learning}
	\begin{itemize}
	\item In inductive learning the learner is given $H$ and $D= \{\langle x_1, f(x_1) \rangle,\ldots,\langle x_n, f(x_n)\rangle\}$, where $f(x_i)$ is the target value for instance $x_i$. Learner must pick $h \in H$ that is consistent with $D$.
    	\item In analytical learning the learner is given $H$, $D$, and a domain theory $B$. Learner must pick $h \in H$ that is consistent with both $D$, and $B$.
    	\item We say that $h$ is consistent with $B$ if \[ \neg (B \rightarrow \neg h) \]
	\end{itemize}
\end{slide}


%%%%%%%%%%%%%%%%%%%%%%%%%%%%%%%%%%%%%%%%%%%%%%%%%%%%%%%%%%%%%%%%%%%%%%%%%%%%%%%%%%%%%%%%%%%%%%%%%%%%%%%%%%%

\begin{slide}{The Inductive Generalisation Problem}
\tiny
	{\bf Given:}
	\begin{itemize}
	\item Instance space $X$
	\item Hypothesis space $H$
	\item Training examples $D$ of some target function $f$. 
	\[ D = \{
	\langle x_{1}, f(x_{1}) \rangle, \dots \langle x_{n}, f(x_{n}) \rangle
	\} \]
	\end{itemize}
	
	{\bf Determine:}
	\begin{itemize}
	\item A hypothesis from $H$ consistent with training examples $D$.
	\end{itemize}
\end{slide}

%%%%%%%%%%%%%%%%%%%%%%%%%%%%%%%%%%%%%%%%%%%%%%%%%%%%%%%%%%%%%%%%%%%%%%%%%%%%%%%%%%%%%%%%%%%%%%%%%%%%%%%%%%%

\begin{slide}{The Analytical Generalisation Problem}
\tiny
	{\bf Given:}
	\begin{itemize}
	\item Instance space $X$
	\item Hypothesis space $H$
	\item Training examples $D$ of some target function $f$. 
	\[ D = \{
	\langle x_{1}, f(x_{1}) \rangle, \dots \langle x_{n}, f(x_{n}) \rangle
	\} \]
	\item {\em Domain theory $B$ for explaining training examples}
	\end{itemize}
	
	{\bf Determine:}
	\begin{itemize}
	\item A hypothesis from $H$ consistent with both the training examples $D$ and domain
	theory $B$.
	\end{itemize}
	
	We say 
	\begin{itemize}
	\item
	$B$ ''explains'' $\langle x, f(x) \rangle$ if $x + B \models f(x)$  
	\item
	$B$ is ''consistent with'' $h$ if  $B \not \models \neg h$
	\end{itemize}
\end{slide}

%%%%%%%%%%%%%%%%%%%%%%%%%%%%%%%%%%%%%%%%%%%%%%%%%%%%%%%%%%%%%%%%%%%%%%%%%%%%%%%%%%%%%%%%%%%%%%%%%%%%%%%%%%%

\begin{slide}{An Analytical Generalisation Problem}
\tiny
	{\bf Given:}
	\begin{itemize}
	\item Instances: pairs of physical objects
	\item Hypotheses:  Sets of Horn clause rules, e.g.,
	\[ SafeToStack(x,y) \gets Volume(x, vx) \wedge Type(y,Box) \]
	
	\item Training Examples: typical example is
	\begin{tabbing}
	\vspace{-.25 in}
	% following sets up tab stops
	xxxx\=xxxxxxxxxxxxxxxxxxxxx\=xxxxxxxxxxxxxxxxxxxx\= \kill 
	$SafeToStack(Obj{\it 1}, Obj{\it 2})$ \\
	\>\hskip-20pt$On(Obj{\it 1},Obj{\it 2})$	\>$Owner(Obj{\it 1}, Fred)$ \\
	\>\hskip-20pt$Type(Obj{\it 1},Box)$	\>$Owner(Obj{\it 2}, Louise)$ \\
	\>\hskip-20pt$Type(Obj{\it 2},Endtable)$ \>$Density(Obj{\it 1},0.3)$ \\
	\>\hskip-20pt$Color(Obj{\it 1},Red)$	\>$Material(Obj{\it 1}, Cardbd)$ \\
	\>\hskip-20pt$\ldots$
	\end{tabbing}
\end{itemize}
\end{slide}

%%%%%%%%%%%%%%%%%%%%%%%%%%%%%%%%%%%%%%%%%%%%%%%%%%%%%%%%%%%%%%%%%%%%%%%%%%%%%%%%%%%%%%%%%%%%%%%%%%%%%%%%%%%

\begin{slide}{An Analytical Generalisation Problem}
\tiny	
\begin{itemize}
 \item 
	Domain Theory:
	\begin{tabbing}
	xxxx\=xxxxxxxxxxxxxxxxxxxxxxxx\=xxxxxxxxxxxxxxxxxxxx\= \kill 
		\>\hskip-16pt$ SafeToStack(x,y) \gets \neg Fragile(y)$ \\
		\>\hskip-16pt$SafeToStack(x,y) \gets Lighter(x,y)$ \\
		\>\hskip-16pt$Lighter(x,y) \gets Wt(x,wx)\wedge Wt(y,wy)\wedge Less(wx,wy)$ \\
		\>\hskip-16pt$\ldots$ \\
	\end{tabbing}
	\end{itemize}
	\vspace{-.1in}
	{\bf Determine:}
	\begin{itemize}
	\item A hypothesis from $H$ consistent with training examples and domain theory.
	\end{itemize}
\end{slide}



%%%%%%%%%%%%%%%%%%%%%%%%%%%%%%%%%%%%%%%%%%%%%%%%%%%%%%%%%%%%%%%%%%%%%%%%%%%%%%%%%%%%%%%%%%%%%%%%%%%%%%%%%%%

\begin{slide}{Learning from Perfect Domain Theories}
\tiny
	\begin{itemize}
 	\item A perfect domain theory is correct and complete.
    \item A domain theory is correct if each of its assertions is a truthful statement about the world.
    \item A domain the is complete wrt target concept and X, if it covers every positive example in the instance space.
    \item So, if we have a perfect domain theory, why do we need to learn?
    \item Chess. Often the theory leads to too many deductions (large breadth) making it impossible to find the optimal strategy. The examples help to focus search.
    \item Perfect domain theories are often unrealistic, but, learning in them is a first step before learning with imperfect theories (next chapter).
    \item Prolog-EBG is an EBL learner. It uses sequential covering.
\end{itemize}

\end{slide}


%%%%%%%%%%%%%%%%%%%%%%%%%%%%%%%%%%%%%%%%%%%%%%%%%%%%%%%%%%%%%%%%%%%%%%%%%%%%%%%%%%%%%%%%%%%%%%%%%%%%%%%%%%%

\begin{slide}{Learning from Perfect Domain Theories}
\tiny
	\texttt{Prolog-EBG}($TargetConcept, Examples, DomainTheory$)
	
	\begin{itemize}
	\item $LearnedRules \gets \{\}$
	\item $Pos \gets$ the positive examples from $Examples$ 
	\item
	for each $PositiveExample$ in $Pos$ that is not covered by $LearnedRules$, do
	\begin{itemize}
	\item[]\hspace*{-.18in}\hskip-12pt{\em 1. Explain:}
	\item $Explanation \gets $ an explanation (proof) in terms of {\em
	DomainTheory} that {\em PositiveExample} satisfies {\em TargetConcept}
	
	\item[]\hspace*{-.18in}\hskip-12pt{\em 2. Analyse:}
	\item 
	$SufficientConditions \gets $ the most general set of features of {\em PositiveExample} that satisfy {\em TargetConcept} according to $Explanation$.
	\end{itemize}	
	\item
	Return $LearnedRules$
	\end{itemize}
\end{slide}

	
\begin{slide}{Learning from Perfect Domain Theories \textbf{ctd}}
\tiny
\begin{itemize}
 \item 
 \begin{itemize}
	\item[]\hspace*{-.18in}\hskip-12pt{\em 3. Refine:}
	\item
	$LearnedRules \gets LearnedRules \ + \ NewHornClause$, where {\em
	NewHornClause} is of the form
	\[TargetConcept \gets  SufficientConditions \]
	\end{itemize}	
	\item
	Return $LearnedRules$
	\end{itemize}
\end{slide}

%%%%%%%%%%%%%%%%%%%%%%%%%%%%%%%%%%%%%%%%%%%%%%%%%%%%%%%%%%%%%%%%%%%%%%%%%%%%%%%%%%%%%%%%%%%%%%%%%%%%%%%%%%%

\begin{slide}{An Explanation }
\tiny
	\begin{itemize}
	 \item Give a proof, using the domain theory, that the (positive) training satisfies the target concept.
	\item Referencing the scenario, the positive example of SafeToStack(o1,o2) can be explained by using the domain theory, as such:
		\begin{enumerate}
		\item Volume(o1,2) $\wedge$ Density(o1,0.3) $\wedge$ Equal(0.6, 2*0.3) $\rightarrow$ Weight(o1,0.6)
		\item Type(o2,endtable) $\rightarrow$  Weight(o2,5)
		\item Weight(o1, 0.6) $\wedge$ LessThan(0.6, 5) $\wedge$ Weight(o2,5) $\rightarrow$  Lighter(o1,o2)
		\item Lighter(o1, o2) $\rightarrow$ SafeToStack(o1,o2) 
		\end{enumerate}
	\item In Prolog-EGB this explanation is generated using backward chaining search, as done by Prolog.
	\item Like Prolog, it halts when it finds a proof
	\end{itemize}
\end{slide}

%%%%%%%%%%%%%%%%%%%%%%%%%%%%%%%%%%%%%%%%%%%%%%%%%%%%%%%%%%%%%%%%%%%%%%%%%%%%%%%%%%%%%%%%%%%%%%%%%%%%%%%%%%%

\begin{slide}{An Explanation }
	\begin{figure}
		\centering
		\includegraphics[scale=0.35]{./../bookps/ebg-safetostack1.epsf}
	\end{figure}
\end{slide}

%%%%%%%%%%%%%%%%%%%%%%%%%%%%%%%%%%%%%%%%%%%%%%%%%%%%%%%%%%%%%%%%%%%%%%%%%%%%%%%%%%%%%%%%%%%%%%%%%%%%%%%%%%%

\begin{slide}{Computing Weakest Preimage of $SafeToStack(Obj1, Obj2)$  }
	\begin{figure}
		\centering
		\includegraphics[scale=0.4]{./../bookps/ebg-safetostack3.epsf}
	\end{figure}
\end{slide}

%%%%%%%%%%%%%%%%%%%%%%%%%%%%%%%%%%%%%%%%%%%%%%%%%%%%%%%%%%%%%%%%%%%%%%%%%%%%%%%%%%%%%%%%%%%%%%%%%%%%%%%%%%%

\begin{slide}{Preimage $SafeToStack(Obj1, Obj2)$  }
\tiny
	\texttt{Regress}({\em Frontier, Rule, Literal, $\theta_{hi}$})
	
	\begin{itemize}
	\item[] {\em Frontier:  literals to be regressed}
	\item[] {\em Rule: A Horn clause}
	\item[] {\em Literal: A literal in Frontier inferred by
	Rule in the explanation}
	\item[] $\theta_{hi}${\em : The substitution that unifies the head of Rule to the
	corresponding literal in the explanation}
	\end{itemize}
ctd..
\end{slide}

%%%%%%%%%%%%%%%%%%%%%%%%%%%%%%%%%%%%%%%%%%%%%%%%%%%%%%%%%%%%%%%%%%%%%%%%%%%%%%%%%%%%%%%%%%%%%%%%%%%%%%%%%%%

\begin{slide}{Preimage $SafeToStack(Obj1, Obj2)$  }
\tiny
	
	\begin{itemize}
	\item[] {\em Returns the set of literals forming the weakest preimage of
	Frontier with respect to Rule}
	
	\begin{itemize}
	\item $head \gets head$ of {\em Rule}
	\item $body \gets body$ of {\em Rule}
	\item $\theta_{hl} \gets$ the most general unifier of $head$ with {\em Literal} such
	that there exists a substitution $\theta_{li}$ for which
	\[ \theta_{li}(\theta_{hl}(head)) = \theta_{hi}(head) \]
	\item Return $\theta_{hl} ( Frontier - head + body)$
	\end{itemize}
	\end{itemize}
\end{slide}
%%%%%%%%%%%%%%%%%%%%%%%%%%%%%%%%%%%%%%%%%%%%%%%%%%%%%%%%%%%%%%%%%%%%%%%%%%%%%%%%%%%%%%%%%%%%%%%%%%%%%%%%%%%

\begin{slide}{ Regressing Literals: An Example }
\tiny 
	\texttt{Regress}({\em Frontier, Rule, Literal, $\theta_{hi}$}) where
	\begin{itemize}
	\item[]{\em Frontier} $= \{ Volume(x,vs), Density(x,dx), Eq(wx,
	times(vx,$$dx)),$ $LessThan($$wx,$$wy)$, $Weight(y,$$wy) \}$
	\item[]{\em Rule} $= Weight(z,5) \gets Type(z,Endtable)$
	\item[]{\em Literal} $= Weight(y,wy)$
	\item[]$\theta_{hi} = \{z / Obj2 \}$
	\begin{itemize}
	\item $head \gets Weight(z,5)$
	\item $body \gets  Type(z,Endtable)$
	\item $\theta_{hl} \gets \{z / y, wy / 5 \}$, where $\theta_{li} = \{y / Obj2 \}$
	\item Return $\{ Volume(x,vs), Density(x,dx), Eq(wx, times(vx,dx)),$ $LessThan(wx,5),$ $Type(y,Endtable) \}$
	\end{itemize}
	\end{itemize}
\end{slide}

%%%%%%%%%%%%%%%%%%%%%%%%%%%%%%%%%%%%%%%%%%%%%%%%%%%%%%%%%%%%%%%%%%%%%%%%%%%%%%%%%%%%%%%%%%%%%%%%%%%%%%%%%%%

\begin{slide}{ Lessons from $SafeToStack$ Example }
	
	\begin{itemize}
	\item Justified generalisation from single positive example
	\item Explanation determines  attribute {\em relevance}
	\item Regression determines sufficient constraints on attributes
	\item Generality of output hypothesis depends on domain theory
	\item Still requires multiple examples (covering algorithm)
	\end{itemize}
\end{slide}


%%%%%%%%%%%%%%%%%%%%%%%%%%%%%%%%%%%%%%%%%%%%%%%%%%%%%%%%%%%%%%%%%%%%%%%%%%%%%%%%%%%%%%%%%%%%%%%%%%%%%%%%%%%

\begin{slide}{Inductive Bias of Prolog-EBG}
	\begin{itemize}
	\item Since all the candidate hypotheses are generated from B it follows that the inductive bias of Prolog-EBG is simply B, right?
    	\item Almost. We also have to consider how it chooses from among the alternative clauses.
    	\item Since it uses sequential covering by growing the Horn clauses we can say that it prefers small sets of Horn clauses.
    	\item So, the inductive bias is B plus a preference for small sets of maximally general Horn clauses.
    \item The inductive bias is largely determined by the input domain theory, not the algorithm.
\end{itemize}
\end{slide}

%%%%%%%%%%%%%%%%%%%%%%%%%%%%%%%%%%%%%%%%%%%%%%%%%%%%%%%%%%%%%%%%%%%%%%%%%%%%%%%%%%%%%%%%%%%%%%%%%%%%%%%%%%%

\begin{slide}{Perspectives on Prolog-EBG }
	\begin{itemize}
	\item Theory-guided generalisation from examples
	\item Example-guided compilation of domain theory
	\item ''Just'' restating what learner already ``knows''
	\end{itemize}
	
	But is it learning?
	\begin{itemize}
	\item Are you learning when you improve at playing chess?
		\begin{itemize} 
		\item Even though optimal play follows from rules of game... 
		\end{itemize}
	\item Are you learning when you sit in math class?
		\begin{itemize} \item Even though those theorems follow from the axioms you've already learned... \end{itemize}
	\end{itemize}
\end{slide}

\end{document}




