\documentclass[%
pdf,
%nocolorBG,
colorBG,
slideColor,
tcrico,
%slideBW,
%draft,
%frames
%azure
%contemporain
%nuancegris
%troispoints
%lignesbleues
%darkblue
%alienglow
%autumn
%default
%gyom
%blends
]{prosper}
\usepackage{amsmath}
%\usepackage[table]{xcolor}
\usepackage{colortbl}
\usepackage{graphicx}
\usepackage{subfigure}
\usepackage{algorithm2e}

\begin{document}

\begin{slide}{Knowledge Based Technologies}
	\textbf{Lecture 6} 
	\newline
	\textbf{An Introduction to Genetic Algorithms}

	\small
	John Moore \& Thomas Collins
\end{slide}


\begin{slide}{Genetic Algorithms: Synopsis}
	\begin{itemize}
	\item Premise
		\begin{itemize}
		\item Evolution worked once (it produced us!), it might work again
		\end{itemize}
	\item Basics
		\begin{itemize}
		\item Pool of solutions
	`	\item  Mate existing solutions to produce new solutions
		\item Mutate current solutions for long-term diversity
		\item Cull population
		\end{itemize}
	\end{itemize}
\end{slide}


\begin{slide}{Originator}
\tiny
\begin{itemize}
\item Developed: USA in the 1970's
\item Early names: J. Holland, K. DeJong, D. Goldberg
\item Typically applied to:
\begin{itemize}
\item Discrete optimization
\end{itemize}
\item Attributed features:
\begin{itemize}
\item Considered somewhat slow to converge on a solution
\item Good heuristic for combinatorial problems
\end{itemize}
\item Special Features:
\begin{itemize}
\item Traditionally emphasizes combining information from good parents (crossover)
\item Many variants, e.g., reproduction models, operators
\end{itemize}
\end{itemize}
\end{slide}


\begin{slide}{Overview}
\begin{itemize}
\item Computing pioneers (especially in AI) looked to natural systems as guiding metaphors.
\item Evolutionary computation
\begin{itemize}
\item Any biologically-motivated computing activity simulating natural evolution
\end{itemize}
\item Genetic Algorithms are one form of this activity
\item Original goals
\begin{itemize}
\item Formal study of the phenomenon of adaptation - John Holland
\item An optimization tool for engineering problems
\end{itemize}
\end{itemize}
\end{slide}


%%%%%%%%%%%%%%%%%%%%%%%%%%%%%%%%%%%%%%%%%%%%%%%%%%%%%%%%%%%%%%%%%%%%%%%%%%%%%%%%%%%%%%%%%%%%%%%%%%%%%

\begin{slide}{ Biological Evolution}  
\begin{itemize} 
\item Lamarck and others:
\begin{itemize} 
	\item Species ''transmute'' over time 
\end{itemize}
\item Darwin and Wallace:
\begin{itemize} \item Consistent, heritable variation among individuals in population
\item Natural selection of the fittest
\end{itemize}
\item Mendel and genetics:
\begin{itemize} \item A mechanism for inheriting traits
\item genotype $\rightarrow$ phenotype mapping
\end{itemize}
\end{itemize}

\end{slide}

%%%%%%%%%%%%%%%%%%%%%%%%%%%%%%%%%%%%%%%%%%%%%%%%%%%%%%%%%%%%%%%%%%%%%%%%%%%%%%%%%%%%%%%%%%%%%%%%%%%%%
\begin{slide}{Why Evolution?}
\begin{itemize}
\item Ability to efficiently guide a search through a large solution space
\item Ability to adapt solutions to changing environments
\item ''Emergent'' behavior is the goal
\end{itemize}
\begin{quote}
\texttt{''The hoped-for emergent behavior is the design of high-quality solutions to difficult problems and the ability to adapt these solutions in the face of a changing environment'' 
}\end{quote}
Melanie Mitchell, An Introduction to Genetic Algorithms
\end{slide}

%%%%%%%%%%%%%%%%%%%%%%%%%%%%%%%%%%%%%%%%%%%%%%%%%%%%%%%%%%%%%%%%%%%%%%%%%%%%%%%%%%%%%%%%%%%%%%%%%%%%%

\begin{slide}{ Approach}  
\begin{itemize}
\item Take a population of candidate solutions to a given problem
\item Use operators inspired by the mechanisms of natural genetic variation
\item Apply selective pressure toward certain properties
\item Evolve a more fit solution 
\end{itemize}

\end{slide}

%%%%%%%%%%%%%%%%%%%%%%%%%%%%%%%%%%%%%%%%%%%%%%%%%%%%%%%%%%%%%%%%%%%%%%%%%%%%%%%%%%%%%%%%%%%%%%%%%%%%%

\begin{slide}{ Terminology  }  
\tiny
\begin{itemize}
\item Abstractions imported from biology
	\begin{itemize}
	\item Chromosomes, Genes, Alleles
	\item Fitness, Selection
	\item Crossover, Mutation
	\end{itemize}
\item GA terminology in the spirit - but not the letter - of  biology
\begin{itemize}
\item GA chromosomes are strings of genes
\item Each gene has a number of alleles; i.e., settings
\item Each chromosome is an encoding of a solution to a problem
\item A population of such chromosomes is operated on by a GA
\end{itemize}
\end{itemize}

\end{slide}

%%%%%%%%%%%%%%%%%%%%%%%%%%%%%%%%%%%%%%%%%%%%%%%%%%%%%%%%%%%%%%%%%%%%%%%%%%%%%%%%%%%%%%%%%%%%%%%%%%%%%

\begin{slide}{ Domain Encoding }
\begin{itemize}
\item A data structure for representing candidate solutions
\item Often takes the form of a bit string 
\item Usually has internal structure; i.e., different parts of the string represent different aspects of the solution)
\end{itemize}
\end{slide}

%%%%%%%%%%%%%%%%%%%%%%%%%%%%%%%%%%%%%%%%%%%%%%%%%%%%%%%%%%%%%%%%%%%%%%%%%%%%%%%%%%%%%%%%%%%%%%%%%%%%%

\begin{slide}{ Domain Encoding }
\tiny 
\begin{itemize}
\item Represent \[ (Outlook = Overcast \vee Rain) \wedge (Wind = Strong) \] by
\begin{center}
\begin{tabular}{cc}
$Outlook$ & $Wind$ \\
011 & 10
\end{tabular}
\end{center}
\item Represent
\[\mbox{IF\ \ } Wind = Strong \mbox{\ \ \ THEN\ \ } PlayTennis=yes \]
by
\begin{center}
\begin{tabular}{ccc}
$Outlook$ & $Wind$ & $PlayTennis$ \\
111 & 10 & 10 \\
\end{tabular}
\end{center}
\end{itemize}
\end{slide}

%%%%%%%%%%%%%%%%%%%%%%%%%%%%%%%%%%%%%%%%%%%%%%%%%%%%%%%%%%%%%%%%%%%%%%%%%%%%%%%%%%%%%%%%%%%%%%%%%%%%%


\begin{slide}{ Crossover}
\begin{itemize}
\item Mimics biological recombination
\item Some portion of genetic material is swapped between chromosomes
\item Typically the swapping produces an offspring
\item Mechanism for the dissemination of ''building blocks'' (schemas)
\end{itemize}
\end{slide}


%%%%%%%%%%%%%%%%%%%%%%%%%%%%%%%%%%%%%%%%%%%%%%%%%%%%%%%%%%%%%%%%%%%%%%%%%%%%%%%%%%%%%%%%%%%%%%%%%%%%%

\begin{slide}{ Mutation}  
\begin{itemize}
\item Selects a random locus - gene location - with some probability and alters the allele at that locus
\item The intuitive mechanism for the preservation of variety in the population
\end{itemize}

\end{slide}

%%%%%%%%%%%%%%%%%%%%%%%%%%%%%%%%%%%%%%%%%%%%%%%%%%%%%%%%%%%%%%%%%%%%%%%%%%%%%%%%%%%%%%%%%%%%%%%%%%%

\begin{slide}{ Fitness}
\begin{itemize}
\item A measure of the goodness of the organism
\item Expressed as the probability that the organism will live another cycle (generation)
\item Basis for the natural selection simulation
\begin{itemize}
\item Organisms are selected to mate with probabilities proportional to their fitness
\end{itemize}
\item Probabilistically better solutions have a better chance of conferring their building blocks to the next generation (cycle)
\end{itemize}
\end{slide}

%%%%%%%%%%%%%%%%%%%%%%%%%%%%%%%%%%%%%%%%%%%%%%%%%%%%%%%%%%%%%%%%%%%%%%%%%%%%%%%%%%%%%%%%%%%%%%%%%%%%%


\begin{slide}{  Operators } 
\begin{figure}
	\centering
	\includegraphics[scale=0.4]{./../bookps/ga-recomb.epsf}
\end{figure}
\end{slide}


\begin{slide}{ A Simple GA}

\begin{algorithm}[H]
Generate initial population \;
\While{not converged}
{
	Calculate the fitness of each member \;
		// simulate another generation \;
		\While{new population is not full}
		{
			Select parents from current population \;
			Perform crossover add offspring to the new population \;
		}
	Merge new population into the current population \;
	Mutate current population \;
}
\end{algorithm}



\end{slide}

%%%%%%%%%%%%%%%%%%%%%%%%%%%%%%%%%%%%%%%%%%%%%%%%%%%%%%%%%%%%%%%%%%%%%%%%%%%%%%%%%%%%%%%%%%%%%%%%%%%

\begin{slide}{How do GAs work} 
\tiny
\begin{itemize}
\item The structure of a GA is relatively simple to comprehend, but the dynamic behavior is complex
\item Holland has done significant work on the theoretical foundations of GAs
\begin{quote}
\texttt{''GAs work by discovering, emphasizing, and recombining good ''building blocks'' of solutions in a highly parallel fashion.''
}\end{quote}
Melanie Mitchell, paraphrasing John Holland

\item  Using formalism
\begin{itemize}
\item Notion of a building block is formalized as a schema
\item Schemas are propagated or destroyed according to the laws of probability 
\end{itemize}
\end{itemize}

\end{slide}

%%%%%%%%%%%%%%%%%%%%%%%%%%%%%%%%%%%%%%%%%%%%%%%%%%%%%%%%%%%%%%%%%%%%%%%%%%%%%%%%%%%%%%%%%%%%%%%%%%%

\begin{slide}{ Schema}
\begin{itemize}
\item A template, much like a regular expression, describing a set of strings
\item The set of strings represented by a given schema characterizes a set of candidate solutions sharing a property
\item This property is the encoded equivalent of a building block
\end{itemize}
\end{slide}

%%%%%%%%%%%%%%%%%%%%%%%%%%%%%%%%%%%%%%%%%%%%%%%%%%%%%%%%%%%%%%%%%%%%%%%%%%%%%%%%%%%%%%%%%%%%%%%%%%%

\begin{slide}{ Example }  
\begin{itemize}
\item \texttt{0} or \texttt{1} represents a fixed bit
\item  Asterisk represents a ''don't care''
\item  \texttt{11****00} is the set of all solutions encoded in 8 bits, beginning with two ones and ending with two zeros
\item Solutions in this set all share the same variants of the properties encoded at these loci
\end{itemize}
\end{slide}

%%%%%%%%%%%%%%%%%%%%%%%%%%%%%%%%%%%%%%%%%%%%%%%%%%%%%%%%%%%%%%%%%%%%%%%%%%%%%%%%%%%%%%%%%%%%%%%%%%%

\begin{slide}{ Schema qualifiers}  
\begin{itemize}
\item Length
\begin{itemize}
	\item The inclusive distance between the two bits in a schema which are furthest apart (the defining length of the previous example is 8)
\end{itemize}
\item Order
\begin{itemize}
\item The number of fixed bits in a schema (the order of the previous example is 4) 
\end{itemize}
\end{itemize}

\end{slide}

%%%%%%%%%%%%%%%%%%%%%%%%%%%%%%%%%%%%%%%%%%%%%%%%%%%%%%%%%%%%%%%%%%%%%%%%%%%%%%%%%%%%%%%%%%%%%%%%%%%

\begin{slide}{ Not just sum of the parts }  
\begin{itemize}
\item GAs explicitly evaluate and operate on whole solutions
\item GAs implicitly evaluate and operate on building blocks
\begin{itemize}
	\item Existing schemas may be destroyed or weakened by crossover
	\item New schemas may be spliced together from existing schema
\end{itemize}
\item Crossover includes no notion of a schema - only of the chromosomes
\end{itemize}

\end{slide}

%%%%%%%%%%%%%%%%%%%%%%%%%%%%%%%%%%%%%%%%%%%%%%%%%%%%%%%%%%%%%%%%%%%%%%%%%%%%%%%%%%%%%%%%%%%%%%%%%%%

\begin{slide}{Why do they work}
\begin{itemize}
 \item Schemas can be destroyed or conserved
\item So how are good schemas propagated through generations?
\begin{itemize}
	\item Conserved - good - schemas confer higher fitness on the offspring inheriting them
	\item Fitter offspring are probabilistically more likely to be chosen to reproduce 
\end{itemize}
\end{itemize}

\end{slide}

%%%%%%%%%%%%%%%%%%%%%%%%%%%%%%%%%%%%%%%%%%%%%%%%%%%%%%%%%%%%%%%%%%%%%%%%%%%%%%%%%%%%%%%%%%%%%%%%%%%

\begin{slide}{ Approximating schema dynamics}  
\tiny
\begin{itemize}
 \item Let $H$ be a schema with at least one instance present in the population at time $t$
\item Let $m(H, t)$ be the number of instances of $H$ at time $t$
\item Let $x$ be an instance of $H$ and $f(x)$ be its fitness
\item The expected number of offspring of $x$ is $\frac{f(x)}{f(pop)}$ (by fitness proportionate selection)
\item To know  $E(m(H, t +1))$ (the expected number of instances of schema H at the next time unit), sum $f(x)/f(pop)$ for all x in $H$
\item GA never explicitly calculates the average fitness of a schema, but schema proliferation depends on its value
\end{itemize}

\end{slide}

%%%%%%%%%%%%%%%%%%%%%%%%%%%%%%%%%%%%%%%%%%%%%%%%%%%%%%%%%%%%%%%%%%%%%%%%%%%%%%%%%%%%%%%%%%%%%%%%%%%


\begin{slide}{ Schema Theorm }  
\tiny
\[E[m(s,t+1)] \geq \frac{\hat{u}(s,t)}{\bar{f}(t)}m(s,t) \left(1 -
p_{c}\frac{d(s)}{l-1}\right) (1 - p_{m})^{o(s)} \]

\begin{itemize}
\item $m(s,t) =$ instances of schema $s$ in pop at time $t$
\item $\bar{f}(t) =$ average fitness of pop. at time $t$
\item $\hat{u}(s,t) =$ ave. fitness of instances of $s$ at time $t$
\item $p_c =$ probability of single point crossover operator
\item $p_m = $ probability of mutation operator
\item $l = $ length of single bit strings
\item $o(s)$ number of defined (non ``*'') bits in $s$
\item $d(s) = $ distance between leftmost, rightmost defined bits in $s$
\end{itemize}
\end{slide}
%%%%%%%%%%%%%%%%%%%%%%%%%%%%%%%%%%%%%%%%%%%%%%%%%%%%%%%%%%%%%%%%%%%%%%%%%%%%%%%%%%%%%%%%%%%%%%%%%%%


\begin{slide}{ Approximating schema dynamics }  
\begin{itemize}
 \item Approximation can be refined by taking into account the operators
\begin{itemize}
\item Schemas of long defining length are less likely to survive crossover
\item Offspring are less likely to be instances of such schemas
\end{itemize}
\item Schemas of higher order are less likely to survive mutation
\item Effects can be used to bound the approximat

\end{itemize}

\end{slide}

%%%%%%%%%%%%%%%%%%%%%%%%%%%%%%%%%%%%%%%%%%%%%%%%%%%%%%%%%%%%%%%%%%%%%%%%%%%%%%%%%%%%%%%%%%%%%%%%%%%


\begin{slide}{ Implications }  
\begin{itemize}
 \item Instances of short, low-order schemas whose average fitness tends to stay above the mean will increase exponentially
\item Changing the semantics of the operators can change the selective pressures toward different types of schemas
\end{itemize}

\end{slide}



%%%%%%%%%%%%%%%%%%%%%%%%%%%%%%%%%%%%%%%%%%%%%%%%%%%%%%%%%%%%%%%%%%%%%%%%%%%%%%%%%%%%%%%%%%%%%%%%%%%
\begin{slide}{ See Also }
\tiny
\begin{itemize}
 \item Genetic algorithms are a particular class of evolutionary algorithms (EA).
\item Evolutionary algorithms are a subset of evolutionary computation (EC), an indepth exposition of the field is outside the scope of this setting however related fields include:
\begin{itemize}
\item Genetic programming
	\begin{itemize}
	\item Specialization of genetic algorithms (GA) where each individual is a computer program.  
 	\item Evolve computer programs that perform a user-defined task.
	\end{itemize}
 \item Grammatical evolution 
\begin{itemize}
	\item Related to GP
\item Addresses the ''combinatorial explosion'' problem.  
	\end{itemize}

\end{itemize}

\end{itemize}
\end{slide}

%%%%%%%%%%%%%%%%%%%%%%%%%%%%%%%%%%%%%%%%%%%%%%%%%%%%%%%%%%%%%%%%%%%%%%%%%%%%%%%%%%%%%%%%%%%%%%%%%%%

\begin{slide}{Resources}
Some GA frameworks:
	\begin{itemize}
	\item ECJ19 
		\begin{itemize}
		\item Java based EC library
		\item http://cs.gmu.edu/~eclab/projects/ecj/
		\end{itemize}
	
	\item JGAP (Java Genetic Algorithms Package)
		\begin{itemize}
		\item http://jgap.sourceforge.net/
		\end{itemize}
	
	\end{itemize}
\end{slide}




\begin{slide}{Resources}
Some GA frameworks:
	\begin{itemize}
	
	\item AI4R (Artificial Intelligence for Ruby)
		\begin{itemize}
		\item A collection of ruby algorithms implementations, covering several Artificial intelligence fields
		\item Includes a GA component
		\item http://ai4r.rubyforge.org/
		\end{itemize}
\item EO
		\begin{itemize}
		\item C++ based EC library
		\item http://cs.gmu.edu/~eclab/projects/ecj/
		\end{itemize}
	\end{itemize}
\end{slide}
%%%%%%%%%%%%%%%%%%%%%%%%%%%%%%%%%%%%%%%%%%%%%%%%%%%%%%%%%%%%%%%%%%%%%%%%%%%%%%%%%%%%%%%%%%%%%%%%%%%


\begin{slide}{Summary}


\begin{itemize}
 
\item Conduct randomized, parallel, hill-climbing search through $H$
\item Approach learning as optimization problem (optimize fitness)
\item Nice feature: evaluation of Fitness can be very indirect
	\begin{itemize}
	\item Consider learning rule set for multistep decision making
\end{itemize}
 \item No issue of assigning credit/blame to indiv. steps
\end{itemize}
\end{slide}



\end{document}




